\documentclass[a4paper,12pt]{book}
\usepackage{notes}
\usepackage{hyperref}
\begin{document}
\begin{titlepage}
\begin{center}
\HRule \\[0.5cm]
\textsc{\huge Cubic Curve}\\[0.5cm]
\HRule \\[0.5cm]
\textsc{\Large Notes on the derivation of a cubic curve}\\[0.5cm]
\vfill
Hannah Michelle Ellis\\[1.0cm]
\today
\end{center}
\end{titlepage}

\frontmatter
\tableofcontents
\listoftables
\listoffigures
\clearpage

\mainmatter
\chapter{Introduction}
\section{Scattering Experiment}
Here we will explore three different definitions of a tensor. Starting from the weakest to the most powerful.

\subsection{A Multidimensional Array}
A tensor can be seen a simple multidimensional array of numbers.

The first and most basic tensor is called a scalar which is an array with zero dimension, also called a rank zero tensor. Scalars are then just numbers. Examples include $1,2,-1,\pi$

\clearpage
\appendix
\chapter{Notation}
Due to having many material properties used in calculations, it is useful to be able to track the type of property the symbol represents. To aid in keeping track the following the following notation (unless stated otherwise) will be used throughout this text.
\begin{table}[h]
  \centering
\begin{tabular}{ p{5cm} | p{5cm} | l }
  Quantity & Description & Example \\
  \hline
  Specific quanity or a quanity per unit mass & Tilde above symbol & $\tilde{n}$ \\
  Quantity per unit length & Dash following symbol & $a^{\prime}$ \\
  Quantity per unit area & Double dash following symbol & $a^{\prime\prime}$ \\
  Quantity per unit volume & Triple dash following symbol & $a^{\prime\prime\prime}$ \\
  Quantity per particle & Bar above symbol & $\bar{a}$ \\
  Quantity per mole & Check above symbol & $\check{a}$ \\
  Quantity per unit time & Dot above the symbol & $\dot{a}$\\
\end{tabular}
\caption{Notation used in this book} \label{table:notation}
\end{table}
In the case of space and time based symbols, you might see two used together to represent a quantity that is both spread through the material as well as through time. For example the flux of particles in a beam would be denoted as $\dot{n}^{\prime\prime}$ as it is the number of particles per unit area per unit time.
\clearpage
\addcontentsline{toc}{chapter}{Index}
\printindex
\end{document}
