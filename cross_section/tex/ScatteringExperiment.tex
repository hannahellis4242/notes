Here we will look into scattering cross sections and how the relate to scattering experiments.

\subsection{Experimental Setup}
A typical experiment usually involves a beam of particles which is incident on some sort of target material as shown in figure \ref{fig:setup}.

\begin{figure}[h]
  \centering
\definecolor{ffqqqq}{rgb}{1.,0.,0.}
\begin{tikzpicture}[line cap=round,line join=round,>=triangle 45,x=0.4cm,y=0.4cm]
\clip(-1.7749275951231995,0.9115241367886239) rectangle (15.089970595106898,13.203217669027278);
\fill[line width=2.pt,color=ffqqqq,fill=ffqqqq,fill opacity=0.1899999976158142] (4.,7.) -- (6.,8.) -- (6.,6.) -- (4.,5.) -- cycle;
\draw [line width=2.pt] (1.,9.)-- (3.,8.);
\draw [line width=2.pt] (3.,3.)-- (1.,4.);
\draw [line width=2.pt] (1.,9.)-- (1.,4.);
\draw [line width=2.pt] (3.,8.)-- (3.,3.);
\draw [line width=2.pt] (3.,3.)-- (7.,5.);
\draw [line width=2.pt] (3.,8.)-- (7.,10.);
\draw [line width=2.pt] (1.,9.)-- (5.,11.);
\draw [line width=2.pt] (7.,10.)-- (5.,11.);
\draw [line width=2.pt] (7.,10.)-- (7.,5.);
\draw [line width=2.pt,color=ffqqqq] (4.,7.)-- (6.,8.);
\draw [line width=2.pt,color=ffqqqq] (6.,8.)-- (6.,6.);
\draw [line width=2.pt,color=ffqqqq] (6.,6.)-- (4.,5.);
\draw [line width=2.pt,color=ffqqqq] (4.,5.)-- (4.,7.);
\draw [->,line width=2.pt] (8.,5.) -- (5.,6.5);
\end{tikzpicture}

\caption{Typical scattering experiment setup. A target with a beam of particles incident on it.} \label{fig:setup}
\end{figure}

\subsubsection{The Target}
The target material will have $n^{\prime\prime\prime}$ target particles per unit volume. If the target material is of some known composition\footnote{Either elemental or some known substance} then we can calculate the number of particles per unit mass of the target material.

\begin{equation}
  \label{eq:spesific number}
\tilde{n} = \frac{N_A}{\check{m}}
\end{equation}

where $\check{m}$ is the mass of a mole of the target material and $N_A$ is the Avogadro constant, which is the number of particles per mole\footnote{In the notation used in this book, we should really use the symbol $\check{n}$ as this is the number of particles per mole of substance. However due to this being a constant with an already agreed upon symbol, namely $N_A$, we stall stick to using that here.}.

\begin{framed}
  An Example: Consider a target made of pure $\ch{^{265}_{92}U}$. The mass per mole is just the atomic weight or $\check{m} = 265 \times 10^{-3}kg$. So to calculate the number of particles per unit mass we just divide $N_A$ by the atomic weight. In this case $\tilde{n}=\frac{N_A}{\check{m}}=-\frac{6.02214076\times10^{23}}{265 \times 10^{-3}}kg^{-1}=2.272505947\times10^{24}kg{-1}$
\end{framed}

This can be linked to the density of the target material by
\begin{equation}
  \label{eq:density}
\mathrm{density} =\rho= m^{\prime\prime\prime} = \frac{n^{\prime\prime\prime}}{\tilde{n}}
\end{equation}
Note that in equation -ref{eq:density} we have used the standard symbol $\rho$ for density. Included is the notation used by this text also. It is more typical to use the density of the material to calculate the number density by multiplying the density and the spesific number together.

\begin{framed}
  Example Continued : The density of $\ch{^{265}_{92}U}$ is $1.91 \times 10^4 kg.m^{-3}$, using this we can calculate the number density of our target.
  $n^{\prime\prime\prime}=\rho \tilde{n} = 1.91\times10^2 kg.m^{-3} \times 2.272505947\times10^{24}kg^{-1} = 4.340486359\times10^{28} m^{-3}$
\end{framed}

\subsubsection{The Beam}
The beam will typically have a known flux, $f$ or using the book notation $\dot{n}^{\prime\prime}$ per unit area per unit time. If the area does not come into play, then the beam rate might be given instead.

\subsection{cross section}
The cross section for the whole target is given by
\begin{equation}
  \label{eq:target_cross_section}
\mathrm{cross\:section} = \frac{\mathrm{rate}}{\mathrm{flux}}
\end{equation}
As you can see, it is a fairly common poke factor type equation. ie, something we can measure (here the rate of a particular interaction) is given by some poke factor times by something we can control (here the beam flux) and we poke the system with the thing we can control. So as you can see, if we double the beam flux then the rate will also double as a response.

\begin{framed}
  An Example: Sticking with our Uranium target from previous examples. If we have a beam of neutons with a flux of $10^{6}m^{-2}.s^{-1}$ incident on the target which results in $10^{2}$ fission events per second, we have a cross section of $\sigma = \frac{\mathrm{rate}}{\mathrm{flux}}=\frac{100s^{-1}}{1000000m^{-2}.s^{-1}}=10^{-4}m^{2}=1cm^{2}$
\end{framed}

When we talk about cross section normally though, we are more conserned with the cross section per target particle $\bar{\sigma}$. Unless the target is a single atom thick, we have to consider the attenuation of the beam through the target material.

\subsection{Beam Attenuation}
As the beam passes through the target material, the particles in the beam will undergo colisions with the target particles. This will lead to the beam flux decreasing as it goes through the material, which will also have to factor into any calculations for the cross section.

Imagine a slice of through the target material perpendicular to the beam at a point a distance $x$ into the material and $\delta x$ thick.
$\dot{N}(x)$ beam particles pass enter through area $A$ per unit time, and $\dot{N}(x+\delta x)$ particles leave through area $A$ per unit time through the back face at $x+\delta x$.
The volume of material which the beam passes through is $\delta V=A \delta x$ which will contain $\delta n = n^{\prime\prime\prime}\delta V$ target particles.
\begin{figure}[h]
  \centering
\begin{tikzpicture}[line cap=round,line join=round,>=triangle 45,x=0.5cm,y=0.5cm]
\clip(-0.8964974115101715,-1.334772354752722) rectangle (9.105952385815135,6.351066633059268);
\draw [line width=0.pt] (0.,0.)-- (0.,4.);
\draw [line width=0.pt] (0.,4.)-- (1.,4.);
\draw [line width=0.pt] (0.,0.)-- (1.,0.);
\draw [line width=0.pt] (1.,4.)-- (1.,0.);
\draw (0,-0.2) node[rotate=90] {$x$};
\draw (1,-0.65) node[rotate=90] {$x+\delta x$};
\draw (0,4.5) node[rotate=90] {$N(x)$};
\draw (1,5) node[rotate=90] {$N(x+\delta x$)};
\end{tikzpicture}

\caption{a cross section of the target material perpendicular to the beam} \label{fig:attenuation}
\end{figure}
The rate of interaction for the slice is $r=\bar{\sigma} \delta n$. This means that
\begin{subequations}
\begin{align}
\dot{N}(x+\delta x) - \dot{N}(x) &=-r\\
\dot{N}(x+\delta x) - \dot{N}(x) &=-\bar{\sigma} \delta n\\
 &=-\bar{\sigma} n^{\prime\prime\prime} \delta V\\
 &=-\bar{\sigma} n^{\prime\prime\prime} A \delta x\\
 \frac{\dot{N}(x+\delta x) - \dot{N}(x)}{A \delta x} &= -\bar{\sigma} n^{\prime\prime\prime}\\
 \frac{\frac{\dot{N}}{A}(x+\delta x) - \frac{\dot{N}}{A}(x)}{\delta x} &= -\bar{\sigma} n^{\prime\prime\prime} \\
\lim_{\delta x \to 0} \frac{f(x+\delta x) - f(x)}{\delta x} &= -\bar{\sigma} n^{\prime\prime\prime}\\
\frac{df}{dx} = -\bar{\sigma} n^{\prime\prime\prime} \label{eq:attenuation}
\end{align}
\end{subequations}

$f(x)$ is the beam flux at the depth $x$ into the target material. Equation \ref{eq:attenuation} has the solution
\begin{equation}
  \label{eq:attenuation_solution}
f(x)=f(0)e^{-\bar{\sigma}n^{\prime\prime\prime}x}
\end{equation}

Where the term $e^{-\bar{\sigma}n^{\prime\prime\prime}x}$ is considered to be the attenuation factor of the target material.
\subsubsection{A note on alternative units}
It might be that for a particular problem, you are not given a target number density etc and instead may be given different units for things. For example you may have the number of particles per square metre or some way of calculating it. In this case we can use that $n^{\prime\prime}=n^{\prime\prime\prime}x$

\begin{framed}
An Example: Going back to our $\ch{^{265}_{92}U}$ taget from earlier. It has a thickness of $m^{\prime\prime}=10^{-1}kg.m^{-2}$. It also has a beam of neutrons incident on it. The total cross section for interactions with neutrons is $\bar{\sigma}_t=2.7002\times10^{-26}m^2$ We can work out the attenuation factor as follows. Firstly we note that we can calculate the spesific number (as we did before). $\tilde{n}=\frac{N_A}{\check{m}}$ where $\check{m}=2.65\time10^{-2}kg.mol^{-1}$. Using this we can get the number of target particles per unit area $n^{\prime\prime}=\tilde{n}m^{\prime\prime}$.
Then we get the attenuation factor by $e^{\bar{\sigma}_t n^{\prime\prime}}=e^{2.7002\times10^{-26}m^2 \frac{6.02214076\times10^{23}}{2.65\times10^{-2}kg}10^{-1}kg.m^{-2}}$
\end{framed}
