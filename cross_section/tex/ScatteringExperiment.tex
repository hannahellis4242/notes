Here we will look into scattering cross sections and how the relate to scattering experiments.

\subsection{Experimental Setup}
A typical experiment usually involves a beam of particles which is incident on some sort of target material as shown in figure \ref{fig:setup}.

\begin{figure}[h]
  \centering
\definecolor{ffqqqq}{rgb}{1.,0.,0.}
\begin{tikzpicture}[line cap=round,line join=round,>=triangle 45,x=0.4cm,y=0.4cm]
\clip(-1.7749275951231995,0.9115241367886239) rectangle (15.089970595106898,13.203217669027278);
\fill[line width=2.pt,color=ffqqqq,fill=ffqqqq,fill opacity=0.1899999976158142] (4.,7.) -- (6.,8.) -- (6.,6.) -- (4.,5.) -- cycle;
\draw [line width=2.pt] (1.,9.)-- (3.,8.);
\draw [line width=2.pt] (3.,3.)-- (1.,4.);
\draw [line width=2.pt] (1.,9.)-- (1.,4.);
\draw [line width=2.pt] (3.,8.)-- (3.,3.);
\draw [line width=2.pt] (3.,3.)-- (7.,5.);
\draw [line width=2.pt] (3.,8.)-- (7.,10.);
\draw [line width=2.pt] (1.,9.)-- (5.,11.);
\draw [line width=2.pt] (7.,10.)-- (5.,11.);
\draw [line width=2.pt] (7.,10.)-- (7.,5.);
\draw [line width=2.pt,color=ffqqqq] (4.,7.)-- (6.,8.);
\draw [line width=2.pt,color=ffqqqq] (6.,8.)-- (6.,6.);
\draw [line width=2.pt,color=ffqqqq] (6.,6.)-- (4.,5.);
\draw [line width=2.pt,color=ffqqqq] (4.,5.)-- (4.,7.);
\draw [->,line width=2.pt] (8.,5.) -- (5.,6.5);
\end{tikzpicture}

\caption{Typical scattering experiment setup. A target with a beam of particles incident on it.} \label{fig:setup}
\end{figure}

\subsubsection{The Target}
The target material will have $n^{\prime\prime\prime}$ target particles per unit volume. If the target material is of some known composition\footnote{Either elemental or some known substance} then we can calculate the number of particles per unit mass of the target material.

\begin{equation}
  \label{eq:spesific number}
\tilde{n} = \frac{N_A}{\check{m}}
\end{equation}

where $\check{m}$ is the mass of a mole of the target material and $N_A$ is the Avogadro constant, which is the number of particles per mole\footnote{In the notation used in this book, we should really use the symbol $\check{n}$ as this is the number of particles per mole of substance. However due to this being a constant with an already agreed upon symbol, namely $N_A$, we stall stick to using that here.}.
