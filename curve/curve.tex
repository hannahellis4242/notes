\documentclass[a5paper,12pt]{article}
\usepackage{notes}
\begin{document}
\begin{titlepage}
\begin{center}
\HRule \\[0.5cm]
\textsc{\huge Kinematics}\\[0.5cm]
\HRule \\[0.5cm]
\textsc{\Large Kinematics for particle physics}\\[0.5cm]
\vfill
Hannah Michelle Ellis\\[1.0cm]
\today
\end{center}
\end{titlepage}
\thispagestyle{empty}

\clearpage
\tableofcontents
\listoftables
\listoffigures
\clearpage

\section{Cubic Curve}

\definecolor{ududff}{rgb}{0.30196078431372547,0.30196078431372547,1}
\begin{tikzpicture}[line cap=round,line join=round,>=triangle 45,x=1cm,y=1cm]
\begin{axis}[
x=1cm,y=1cm,
axis lines=middle,
ymajorgrids=true,
xmajorgrids=true,
xmin=-2.551469804209948,
xmax=9.563014982994765,
ymin=6.41691509584807,
ymax=12.065199779700474,
xtick={-2,-1,...,9},
ytick={6.5,7,...,12},]
\clip(-2.551469804209948,6.41691509584807) rectangle (9.563014982994765,12.065199779700474);
\draw[line width=2pt, smooth,samples=100,domain=0:1] plot[parametric] function{-3.72*t**(3)+5.53*t**(2)+1.19*t+1,1.3599999999999994*t**(3)-2.0599999999999987*t**(2)+1.6999999999999993*t+9};
\draw [line width=2pt] (1,9)-- (2.19,10.7);
\draw [line width=2pt] (4,10)-- (2.91,8.34);
\begin{scriptsize}
\draw [fill=ududff] (1,9) circle (2.5pt);
\draw[color=ududff] (1.0767777369243494,9.15471423241602) node {$A$};
\draw [fill=ududff] (4,10) circle (2.5pt);
\draw[color=ududff] (4.085072616968474,10.15485636114912) node {$B$};
\draw [fill=ududff] (2.19,10.7) circle (2.5pt);
\draw[color=ududff] (2.2760304255905876,10.852797271128334) node {$C$};
\draw [fill=ududff] (2.91,8.34) circle (2.5pt);
\draw[color=ududff] (2.9874515120875093,8.492749658002743) node {$D$};
\end{scriptsize}
\end{axis}
\end{tikzpicture}

The general vector equation for a cubic curve is
\begin{equation}\label{eq:curve}
\mathbf{r}=\lambda^3\mathbf{a}+\lambda^2\mathbf{b}+\lambda\mathbf{c}+\mathbf{d}
\end{equation}
with the derivative being
\begin{equation}\label{eq:derivative_curve}
\frac{d\mathbf{r}}{d\lambda}=3\lambda^2\mathbf{a}+2\lambda\mathbf{b}+\mathbf{c}
\end{equation} 
Boundary conditions
$\lambda=0 , r = A , \frac{d r}{d\lambda}=C - A$ and $\lambda=1 , r = B , \frac{d r}{d\lambda}=B - D$
\begin{align}\label{eq:bc_equation}
A &= d \nonumber \\
C-A &= c \nonumber \\
B &= a + b + c + d \nonumber \\
B-D &= 3a + 2 b + c
\end{align}

\begin{equation}
\left[ \begin{matrix}
1 & 0 & 0 & 0 \\
-1 & 0 & 1 & 0 \\
0 & 1 & 0 & 0 \\
0 & 1 & 0 & -1 
\end{matrix} \right]
\left( \begin{matrix} A \\ B\\ C \\D\end{matrix}\right) = \left[ \begin{matrix}
0 & 0 & 0 & 1 \\
0 & 0 & 1 & 0 \\
1 & 1 & 1 & 1 \\
3 & 2 & 1 & 0 
\end{matrix} \right]
\left( \begin{matrix} a \\ b\\ c \\d\end{matrix}\right)
\end{equation}

\begin{equation}
\left[ \begin{matrix}
2 & 1 & -2 & 1 \\
-3 & -2 & 3 & -1 \\
0 & 1 & 0 & 0 \\
1 & 0 & 0 & 0 
\end{matrix} \right]
\left[ \begin{matrix}
1 & 0 & 0 & 0 \\
-1 & 0 & 1 & 0 \\
0 & 1 & 0 & 0 \\
0 & 1 & 0 & -1 
\end{matrix} \right]
\left( \begin{matrix} A \\ B\\ C \\D\end{matrix}\right) = 
\left( \begin{matrix} a \\ b\\ c \\d\end{matrix}\right)
\end{equation}

\begin{equation}
\left( \begin{matrix} a \\ b\\ c \\d\end{matrix}\right)=\left[ \begin{matrix}
1 & -1 & 1 & -1 \\
-1 & 2 & -2 & 1 \\
-1 & 0 & 1 & 0 \\
1 & 0 & 0 & 0 
\end{matrix} \right]
\left( \begin{matrix} A \\ B\\ C \\D\end{matrix}\right) 
\end{equation}

\begin{align}\label{eq:solution}
a &= A-B+C-D \nonumber \\
b &= 2(B-C)+D-A \nonumber \\
c &= C-A \nonumber \\
d &= A
\end{align}

\clearpage
\addcontentsline{toc}{chapter}{Index}
\printindex
\end{document}