\documentclass[12pt,a5paper]{article}
\usepackage{notes}

\title{Transistor Circuits}
\date{\today}
\author{Hannah Ellis}
\begin{document}
\maketitle
\tableofcontents
\section{Basic Common Emitter circuit}
\begin{figure}[h]
	\centering
	\begin{circuitikz}
		\draw (0,0) node[ground]{};
		\draw (0,1.0) node[npn](Q1){Q1};
		\draw (Q1.C) -- (0,2.0);
		\draw (Q1.E) -- (0,-0.0);
		\draw (Q1.B) -- (-1,1.0);
		\draw (0,4.0) to[R,l=$R_c$,i=$I_c$] (0,2.0);
		\draw (-1.5,1) -- (-1,1);
		\draw (-1.5,4.0) to[R,l=$R_b$,i=$I_b$] (-1.5,1.0);
		\draw (0,4.0) -- (-1,4.0);
		\draw (-3,-0.0) to[V,l=$V_{in}$] (-3,4.0);
		\draw (-3,-0.0) node[ground]{};
		\draw (-3,4.0) -- (-1,4.0);
	\end{circuitikz}
\caption{Basic Common Emitter Circuit\label{fig:BCEC}}
\end{figure}
For a transistor in forward active mode, we can use the approximation
\begin{equation}
	I_c = \beta I_b
\end{equation} 
We can use the fact that the voltage drop across a resistor divided by it's resistance is the current that flows through it to get
\begin{equation}
	\frac{V_{in}-V_{ce}}{R_c} = \beta \frac{V_{in}-V_{be}}{R_b}
\end{equation} 
\end{document}
