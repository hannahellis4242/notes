Here we will solve the system of eqations given in section \ref{sec:Gravity}.
\subsection{Conservation of angular momentum revisited}
Firstly we note that
\begin{equation}
\frac{d}{dt}\left(r^2 \frac{d\theta}{dt}\right)=r^2 \frac{d^2\theta}{dt^2} + 2r\frac{dr}{dt}\frac{d\theta}{dt}
\end{equation}
is the same as eqation \ref{eq:angularAcceleration} except for the $r$ term in front of everything.
\begin{equation}
\frac{1}{r}\frac{d}{dt}\left(r^2 \dot{\theta}\right)=r\ddot{\theta} + 2\dot{r}\dot{\theta}=0\label{eq:Solution:constAngular}
\end{equation}
using equation \ref{eq:AngularMomentumInTermsOfrandomega}
\begin{equation}
r^2 \dot{\theta}=\frac{l}{m} \label{eq:Solution:const}
\end{equation}
and is a constant of the motion. I.e. this is a restatement of the conservation of angular momentum.
Using above we can get $\omega$
\begin{equation}
\omega=\frac{l}{mr^2}
\end{equation}
Now if we let $r=\frac{1}{u}$ we find that
\begin{align}
\frac{dr}{dt}&=\frac{du^{-1}}{du}\frac{du}{dt} \nonumber \\
&=-\frac{1}{u^2}\frac{du}{dt} \nonumber \\
&=-r^2\frac{du}{dt} \nonumber \\
&=-\left(r^2\frac{d\theta}{dt}\right)\frac{du}{d\theta} 
\end{align}
Notice that the brakects are the same as in equation \ref{eq:Solution:constAngular} and so is constant.
Now take the derivitive again with respect to time 
\begin{align}
\frac{d^2r}{dt^2}&=-\frac{d}{dt}\left(r^2\frac{d\theta}{dt}\right)\frac{du}{d\theta} - \left(r^2\frac{d\theta}{dt}\right)\frac{d}{dt}\frac{du}{d\theta} \nonumber \\
&=- \left(r^2\frac{d\theta}{dt}\right)\frac{d\theta}{dt}\frac{d}{d\theta}\frac{du}{d\theta} \nonumber \\
&=- r^2\left(\frac{d\theta}{dt}\right)^2\frac{d^2u}{d\theta^2} \label{eq:Solution:radial}
\end{align}
Using equation \ref{eq:Solution:radial} in equation \ref{eq:radialAcceleration} and continuing with the change of variables gives
\begin{align}
-\frac{GM}{r^2}&=-r^2\left(\frac{d\theta}{dt}\right)^2\frac{d^2u}{d\theta^2} - r\left(\frac{d\theta}{dt}\right)^2 \nonumber \\
GM&=r^4\left(\frac{d\theta}{dt}\right)^2\frac{d^2u}{d\theta^2} + r^3\left(\frac{d\theta}{dt}\right)^2 \nonumber \\
GM&=\left[r^2\frac{d\theta}{dt}\right]^2\frac{d^2u}{d\theta^2} + \frac{1}{r}\left[r^2\frac{d\theta}{dt}\right]^2 \nonumber \\
GM&=\left[r^2\frac{d\theta}{dt}\right]^2\left[\frac{d^2u}{d\theta^2} + u\right] \label{eq:Solution:ode}
\end{align}
making the subsitution in equation \ref{eq:Solution:ode} for the first set of brackets on the right hand side with equation \ref{eq:Solution:const} gives us
\begin{align}
GM&=\left(\frac{l}{m}\right)^2\left[\frac{d^2u}{d\theta^2} + u\right] \nonumber \\
\frac{GMm^2}{l^2}&=\frac{d^2u}{d\theta^2} + u
\end{align}

which leads to the solution being
\begin{align}
u(\theta)&=\frac{GMm^2}{l^2}-c\cos \theta \nonumber \\
r(\theta)&=\frac{1}{\frac{GMm^2}{l^2}-c\cos \theta} \label{eq:Solution:uglySolution}
\end{align}

equation \ref{eq:Solution:uglySolution} is a form of the polar equation for an ellipse. Rearanging to the more usual polar form will give us the semi-major axis and the eccentricity of the obit. Let $\gamma=\frac{l^2}{GMm^2}$

\begin{align}
r(\theta)&=\frac{1}{\frac{1}{\gamma}-c\cos \theta} \nonumber \\
r(\theta)&=\frac{\gamma}{1-\frac{c}{\gamma}\cos \theta} \label{eq:Solution:neatSolution}
\end{align}

\begin{framed}
\subsection{Summary}
The equations of motion under gravity are
\begin{align*}
-\frac{GM}{r^2} &=\ddot{r} - r \omega^2 \\
0 &=r \ddot{\theta} + 2 \dot{r} \dot{\theta}
\end{align*}
Which has a solution for $r$ in terms of $\theta$ which is
\begin{equation*}
r(\theta)=\frac{\gamma}{1-\frac{c}{\gamma}\cos \theta}
\end{equation*}
where $\gamma=\frac{l^2}{GMm^2}$. This is the equation for an ellipse in polar coordinates and so shows that orbits are elliptial.
\end{framed}