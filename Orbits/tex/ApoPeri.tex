Starting with the equation of an ellipse in polar form
\begin{equation} \label{eq:ellipse}
r(\theta)=\frac{A(1+e)(1-e)}{1-e \cos \theta}
\end{equation}

When $r$ is maximum then the particle is at the apoasis\footnote{the furthest point from the parent body} of it's orbit. This occurs when $\cos \theta=1$ i.e. when $\theta=0$.
We stall call the radius at apoasis $a$ so using equation\ref{eq:ellipse} with $\cos \theta=1$
\begin{equation} \label{eq:apoasis}
a=\frac{A(1+e)(1-e)}{1-e}=A(1+e)
\end{equation}
Also we can find the periapsis\footnote{the closest point from the parent body} when $r$ is minimum. This occurs when $\cos \theta=-1$ i.e. when $\theta=\pi$. Again using equation \ref{eq:ellipse} with $\cos \theta=-1$
\begin{equation} \label{eq:periapsis}
p=\frac{A(1+e)(1-e)}{1+e}=A(1-e)
\end{equation}

We can sum the equations \ref{eq:apoasis} and \ref{eq:periapsis} above to give 
\begin{align}
a+p&=A+eA+A-eA \nonumber \\
a+p&=2A \nonumber \\
A&=\frac{a+p}{2}\label{eq:ApoPeri:semiMajorAxis}
\end{align}

This also follows from the geometry of an ellipse.

finally taking the difference of the equations \ref{eq:apoasis} and \ref{eq:periapsis} above gives
\begin{align}
a-p&=A+eA-A+eA \nonumber \\
a-p&=2eA \nonumber \\
e&=\frac{a-p}{2A} \nonumber \\
e&=\frac{a-p}{a+p} \label{eq:ApoPeri:ecentricity}
\end{align}

Using the relationship for the focus and the ecentricity
\begin{align}
f&=eA\nonumber \\
&=\frac{a-p}{a+p}\frac{a+p}{2} \nonumber \\
&=\frac{a-p}{2}
\end{align}

\subsection{Relationship between apoasis and periapsis}
Comparing equation \ref{eq:Solution:neatSolution} to the equation for an elipse in polar form
\begin{align}
r=\frac{A(1-e^2)}{1-e\cos\theta} \label{eq:Solution:ellipse}\\
\Rightarrow A(1-e^2)=\gamma
\end{align}
using equation \ref{eq:ApoPeri:semiMajorAxis} and equation \ref{eq:ApoPeri:ecentricity} we get
\begin{align}
\gamma&=\frac{a+p}{2}\left(1-\frac{(a-p)^2}{(a+p)^2}\right) \nonumber \\
&=\frac{(a+p)^2-(a-p)^2}{2(a+p)} \nonumber \\
&=\frac{a^2+2ap+p^2-a^2+2ap-p^2}{2(a+p)} \nonumber \\
&=\frac{2ap}{a+p} \label{eq:ApoPeri:gamma}
\end{align}

\begin{framed}
\subsection{Summary}
The apoasis $a$ and periapsis $p$ radius is given by
\begin{align*}
a&=A(1+e) \\
p&=A(1-e)
\end{align*}

where $A$ is the semi-major axis of the ellipse and $e$ is the ecentricity. $A$ , $e$ and $f$ can be found in terms of $a$ and $p$ by the following
\begin{align*}
A&=\frac{a+p}{2} \\
e&=\frac{a-p}{a+p} \\
f&=\frac{a-p}{2}
\end{align*}

The relationship between the apoasis and periapsis is given by
\begin{equation*}
\gamma=\frac{2ap}{a+p}
\end{equation*}
where $\gamma=\frac{l^2}{GMm^2}$
\end{framed}