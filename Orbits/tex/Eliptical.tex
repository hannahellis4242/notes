Unlike circular orbits, elliptical orbits do not have a constant radius and so we don't see the same simplification as we do for circular orbits\footnote{that $\ddot{r}=0$}. However we already have a relationship bettween $\gamma$ and the apses, equation \ref{eq:ApoPeri:gamma}
\begin{equation}
\gamma=\frac{2ap}{a+p}
\end{equation}

\subsection{Radial Acceleration}
However we can calculate the radial acceleration at the apses by using equation \ref{eq:radialAcceleration}
\begin{align}
\ddot{r}&=r\omega^2-\frac{GM}{r^2}\nonumber \\
\ddot{r}&=\frac{1}{r^3}\left(\frac{l}{m}\right)^2-\frac{GM}{r^2} \nonumber \\
\ddot{r}&=GM\left(\frac{\gamma}{r^3}-\frac{1}{r^2}\right)\label{eq:Elliptical:radialAcceleration}
\end{align}

\subsubsection{Radial Acceleration at Apoapsis}
At apoapsis $r=a$. Substituting this into equation \ref{eq:Elliptical:radialAcceleration} gives
\begin{align}
\ddot{r}(a)&=GM\left(\frac{\gamma}{a^3}-\frac{1}{a^2}\right)
\end{align}

\subsection{Angular momentum}
Since we already have a relationship bettween $\gamma$ and the apses, the angular momentum is easy to find from $\gamma$
\begin{equation}
\frac{l}{m}=\sqrt{GM\frac{2ap}{a+p}}
\end{equation}
\subsection{Period}
Again the period of an orbit should be as simple as using the link between angular momentum and areal velocity and knowing the area of an ellipse
\begin{equation}
\pi AB=\frac{1}{2}\frac{l}{m}T
\end{equation}
substituting in the relationship between $A$,$B$ and the apses.
\begin{subequations}
\begin{align}
A&=\frac{a+p}{2}\\
B&=\sqrt{ap}
\end{align}
\end{subequations}
gives
\begin{align}
\pi \sqrt{ap}\frac{a+p}{2}&=\frac{1}{2}\frac{l}{m}T \nonumber \\
\pi \sqrt{ap}(a+p)&=\sqrt{GM\frac{2ap}{a+p}}T \nonumber \\
T &= \frac{2\pi}{\sqrt{GM\left(\frac{2}{a+p}\right)^2}} \nonumber \\
T &=\frac{2\pi}{\sqrt{GM}}\left(\frac{a+p}{2}\right)^\frac{3}{2}
\end{align}

\subsection{Energy}
Again we use the energy equation \ref{eq:Gravity:TotalEnergy}, and rearranging to get energy per unit mass
\begin{equation}
\frac{E}{m}=\frac{1}{2} \dot{r}^2 +\frac{1}{2r^2} \left(\frac{l}{m}\right)^2- \frac{GM}{r}
\end{equation}
although our orbit isn't circular, there are two points in the orbit where $\dot{r}=0$ allowing us to drop the first term. Since energy is conservered, the energy will be the same for all points along the orbit. The two points where $\dot{r}=0$ are at the apopsis and periapsis. So substituting $r=a$ gives
\begin{align}
\frac{E}{m}&=\frac{1}{2a^2} \left(\frac{l}{m}\right)^2- \frac{GM}{a}\nonumber \\
\frac{E}{m}&=\frac{GM}{2a^2} \gamma- \frac{GM}{a}\nonumber \\
\frac{E}{m}&=GM\left[\frac{2ap}{2a^2(a+p)} - \frac{1}{a}\right] \nonumber \\
\frac{E}{m}&=GM\left[\frac{2ap-2a(a+p)}{2a^2(a+p)}\right] \nonumber \\
\frac{E}{m}&=GM\left[\frac{a^2}{a^2(a+p)}\right] \nonumber \\
\frac{E}{m}&=\frac{GM}{2}\frac{1}{\frac{a+p}{2}}
\end{align}
Which again shows that for elliptical orbits we get the same pattern as circular orbits, except we take the average of the orbital radius in place of $r$.
\begin{framed}
\subsection{Summary}
\begin{align*}
\gamma&=\frac{2ap}{a+p}\\
\frac{l}{m}&=\sqrt{GM\frac{2ap}{a+p}}\\
T &=\frac{2\pi}{\sqrt{GM}}\left(\frac{a+p}{2}\right)^\frac{3}{2}\\
\frac{E}{m}&=\frac{GM}{2}\frac{1}{\frac{a+p}{2}}
\end{align*}
\end{framed}