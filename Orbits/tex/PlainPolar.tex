In this section we will derive the equations of motion for an object acted upon by gravity in plane
polar coordinates. To do this we will first consider plane polar coordinates themselves and the derivatives of the position vector within the coordinate system.
\begin{figure}[h]
  \centering
\definecolor{qqwuqq}{rgb}{0,0.39,0}
\begin{tikzpicture}[line cap=round,line join=round,>=triangle 45,x=1.0cm,y=1.0cm]
\draw[->,color=black] (-2,0) -- (4,0);
\foreach \x in {-2,-1,1,2,3}
\draw[shift={(\x,0)},color=black] (0pt,-2pt);
\draw[->,color=black] (0,-0.5) -- (0,4);
\clip(-2,-0.5) rectangle (4,4);
\draw [shift={(0,0)},color=qqwuqq,fill=qqwuqq,fill opacity=0.1] (0,0) -- (0:0.75) arc (0:60:0.75) -- cycle;
\draw [shift={(0,0)},color=qqwuqq,fill=qqwuqq,fill opacity=0.1] (0,0) -- (90:0.75) arc (90:150:0.75) -- cycle;
\draw [->] (0,0) -- (1,1.73);
\draw [->] (1,1.73) -- (1.5,2.6);
\draw [->] (0,0) -- (-0.87,0.5);
\draw [->] (1,1.73) -- (0.13,2.23);
\begin{scriptsize}
\draw[color=black] (0.7,0.85) node {$\textbf{r}$};
\draw[color=qqwuqq] (0.5,0.2) node {$\theta$};
\draw[color=qqwuqq] (-0.25,0.5) node {$\theta$};
\draw[color=black] (1.1,2.2) node {$\hat{\textbf{r}}$};
\draw[color=black] (0.6,2.2) node {$\hat{\bm{\theta}}$};
\end{scriptsize}
\end{tikzpicture}
 
\caption{Position and unit vectors in plane polar coordinates} \label{fig:PolarCoordinates}
\end{figure}

\subsection{Polar unit vectors}
The unit vectors of plane polar coordinates can be determined from the standard unit vectors as
\begin{align}
\hat{\bm{r}} &= \uveci \cos( \theta ) + \uvecj \sin( \theta ) \label{eq:unitrad} \\
\hat{\bm{\theta}}  &= - \uveci \sin( \theta ) + \uvecj \cos( \theta ) \label{eq:unitangle}
\end{align}
Where $r$ is the distance from the origin to the position or alternatively the length of the position vector, and $\theta$ is the angle between $\uveci$ and $\bm{r}$.
Since the unit vectors $\hat{\bm{r}}$ and $\hat{\bm{\theta}}$ depend on $\theta$ which changes with time, the unit vectors also change with time. We will begin by looking at the derivative of $\hat{\bm{r}}$ with respect to time.

\begin{equation} \label{eq:radvel}
\frac{d {\hat{\bm{r}}}}{d t} = \frac{ d \theta}{d t}\frac{d}{d \theta} \hat{\bm{r}} =  \frac{ d \theta}{d t} \left[ -\uveci \sin( \theta ) + \uvecj \cos( \theta ) \right] = \frac{ d \theta}{d t} \hat{\bm{\theta}}
\end{equation}

where we have used equation \ref{eq:unitangle} to replace the typical Cartesian coordinate unit vectors.

\begin{equation}\label{eq:angvel}
\frac{d {\hat{\bm{\theta}}}}{d t}=\frac{d \theta}{d t} \frac{d}{dt}{\hat{\bm{\theta}}}=\frac{d \theta}{d t}  \left[ - \uveci \cos( \theta ) - \uvecj \sin( \theta ) \right]=- \frac{d \theta}{d t} \hat{\bm{r}}
\end{equation}

 Since $\frac{d {\hat{\bm{r}}}}{d t}$ too depends on $\theta$ so is also time dependent.

\begin{equation} \label{eq:radvel}
\frac{d^2 {\hat{\bm{r}}}}{d t^2} = \frac{d}{dt}\left[ \frac{ d \theta}{d t} \hat{\bm{\theta}} \right]=\frac{d^2 \theta}{d t^2}\hat{\bm{\theta}}+\frac{d \theta}{d t} \frac{d \hat{\bm{\theta}}}{dt}=\frac{d^2 \theta}{d t^2}\hat{\bm{\theta}}-\left(\frac{d \theta}{d t} \right)^2  \hat{\bm{r}} 
\end{equation}

\begin{framed}
\subsubsection{Polar unit vectors summary}
The polar coordinate unit vectors are
\begin{align*}
\hat{\bm{r}} &= \uveci \cos( \theta ) + \uvecj \sin( \theta ) \\
\hat{\bm{\theta}}  &= - \uveci \sin( \theta ) + \uvecj \cos( \theta )
\end{align*}
With the first derivatives being
\begin{align*}
\frac{d {\hat{\bm{r}}}}{d t} &=  \omega \hat{\bm{\theta}} \\
\frac{d {\hat{\bm{\theta}}}}{d t} &=-\omega \hat{\bm{r}}
\end{align*}
And the second derivative of $\hat{\bm{r}}$ being
\begin{equation*}
\frac{d^2 {\hat{\bm{r}}}}{d t^2} =\frac{d \omega}{d t}\hat{\bm{\theta}}-\omega^2  \hat{\bm{r}} 
\end{equation*}
with $\omega = \frac{d \theta}{d t}$
\end{framed}

\subsection{Acceleration in polar coordinates}
Here we will calculate the acceleration in terms of the polar unit vectors. Starting with the position vector
\begin{equation}
\textbf{r}= r \hat{\textbf{r}}
\end{equation}
This is the length of the position vector in the direction of $\hat{\bm{r}}$. The first derivative will be the velocity of the position vector in terms of the polar unit vectors
\begin{equation}\label{eq:vel}
\frac{d \bm{r}}{d t} = \frac{d r}{d t} \hat{\bm{r}} + r \frac{d \hat{\bm{r}}}{d t}=\frac{d r}{d t} \hat{\bm{r}} + r \omega \hat{\bm{\theta}}
\end{equation}
The velocity is the sum of the radial velocity in the radial direction $\dot{r}$ and the angular part of the velocity $r \omega$ in the angular direction.

The acceleration
\begin{align}\label{eq:acceleration}
\frac{ d^2 \bm{r}}{ d t^2} &= \frac{ d^2 r}{d t^2} \hat{\bm{r}} + \frac{d r}{d t} \frac{d \hat{\bm{r}}}{d t} +\frac{d r}{d t} \omega \hat{\bm{\theta}} +r \frac{d \omega}{d t} \hat{\bm{\theta}}+r \omega \frac{d \hat{\bm{\theta}}}{d t} \nonumber \\
&=  \frac{ d^2 r}{d t^2} \hat{\bm{r}} + \frac{d r}{d t} \omega \hat{\bm{\theta}} +\frac{d r}{d t} \omega \hat{\bm{\theta}} +r \frac{d \omega}{d t} \hat{\bm{\theta}}-r \omega^2 \hat{\bm{r}} \nonumber \\
&=\left[  \frac{ d^2 r}{d t^2} -r \omega^2 \right] \hat{\bm{r}} + \left[ 2 \frac{d r}{d t} \omega + r \frac{ d \omega}{d t} \right] \hat{\bm{\theta}}
\end{align}

\begin{framed}
\subsubsection{Acceleration in polar coordinates summary}
The position vector
\begin{equation*}
\bm{r}=r \hat{\bm{r}}
\end{equation*}
Velocity
\begin{equation*}\label{eq:vel}
\frac{d \bm{r}}{d t} =\frac{d r}{d t} \hat{\bm{r}} + r \omega \hat{\bm{\theta}}
\end{equation*}
Acceleration
\begin{equation*}
\frac{ d^2 \bm{r}}{ d t^2} =\left[  \frac{ d^2 r}{d t^2} -r \omega^2 \right] \hat{\bm{r}} + \left[ 2 \frac{d r}{d t} \omega + r \frac{ d \omega}{d t} \right] \hat{\bm{\theta}}
\end{equation*}
\end{framed}