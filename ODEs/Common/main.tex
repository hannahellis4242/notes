\documentclass[a5paper,10pt]{book}
\usepackage{notes}
\usepackage{hyperref}
\begin{document}

\title{Some Common Ordinary Differential Equations}
\author{Hannah Ellis}
\date{\today}

\maketitle
\tableofcontents
\clearpage

\chapter{Linear Ordinary Differential Equations}
A \href{https://en.wikipedia.org/wiki/Linear_differential_equation}{linear ordinary differential equation} is of the form
\begin{equation}
	b(x) = \sum_0^n a_{i}(x) \frac{d^i y}{dx^i} = a_0(x)y + a_1(x)\frac{dy}{dx}+a_2(x)\frac{d^2 y}{dx^2}+\ldots
\end{equation}

When all of the $a_i(x)$ are constants and $b(x)$ is zero we get the homogeneous linear ordinary differential equation with constant coefficients which looks like this
\begin{equation}
	0 = \sum_0^n a_{i} \frac{d^i y}{dx^i} = a_0y + a_1\frac{dy}{dx}+a_2\frac{d^2 y}{dx^2}+\ldots
\end{equation}

\section{First Order Homogenous Constant Linear Ordinary Differential Equation}

In the case of first order, $n=1$ and we get
\begin{equation}
	a_1\frac{dy}{dx}+ a_0 y = 0
\end{equation}
which can be rearranged into the more common form
\begin{equation}
\frac{dy}{dx} = k y
\end{equation}
When using the trial equation $y=exp(\alpha x)$ we get the general solution of

\begin{equation}
y=c \exp(kx)
\end{equation}
where $c$ is the integration constant.

\section{Second Order Homogenous Constant Linear Ordinary Differential Equation}

In the case of second order, $n=2$ and we get
\begin{equation}
	a_2 \frac{d^2 y}{dx^2}+a_1\frac{dy}{dx}+ a_0 y = 0
\end{equation}

\subsection{In the case that $a_1=0$}
In the case that $a_1=0$ we get the more familiar form as
\begin{equation}
\frac{d^2 y}{dx^2}= k y 
\end{equation}

Using our typical trial equation $y=exp(\alpha x)$ gives
\begin{equation}
\alpha=\sqrt{k}
\end{equation}

\subsubsection{When $k>0$}
The solution becomes
\begin{align}
	y(x) &= A \exp(\alpha x)+B\exp(-\alpha x) \nonumber \\
	&= \tilde{A} \cosh(\alpha x) + \tilde{B} \sinh(\alpha x)
\end{align}
Where we have used the result from \href{../../Maths/ChangeOfBasisOfTypicalSolutions.md}{here} to change the form.

\subsubsection{When $k=0$}
In the case that $k=0$ the differential equation changes to be
\begin{equation}
	\frac{d^2 y}{dx^2}= 0 
\end{equation}
which has the solution
\begin{equation}
y(x)=mx+c
\end{equation}

\subsubsection{When $k<0$}
When $k<0$ then $\alpha$ is imaginary, so letting $\beta = i\sqrt{-k}$ gives a solution of the form
\begin{align}
	y(x) &= A \exp(\beta x)+B\exp(-\beta x) \nonumber \\
	&= \tilde{A} \cos(\beta x) + \tilde{B} \sin(\beta x)
\end{align}

\end{document}
