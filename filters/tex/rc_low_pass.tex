\subsection{ Filter Attenuation }
A simple low pass filter can be formed from a resistor and a capacitor in series.

\begin{figure}[h]
  \centering
\begin{circuitikz}
\draw (0,0) to [short,*-,l=$V_{in}$] (1,0)
  to [european resistor, l=$R$] (3,0)
  to [C,l=$C$] (3,-2)
  to (3,-2) node[ground]{};
\draw (3,0) to [short,-*,l=$V_{out}$] (5,0);
\end{circuitikz}
\caption{Circuit diagram for a RC low pass filter.} \label{fig:RC_low_pass_circuit}
\end{figure}
We can see from figure \ref{fig:RC_low_pass_circuit}, that the output voltage is
the same as the voltage across the capacitor. So we have
\begin{align}
V_{in} &= V_{R} + V_{C} \nonumber \\
       &= IR + V_{out}
\end{align}
Using the fact that the current through the RC section of the circuit is given
by
\begin{equation}
  I = \frac{V_{in}}{R - i X_C}
\end{equation}
Leading to the output voltage being
\begin{align}
V_{out} &= V_{in} - IR \nonumber \\
&= V_{in} - \frac{V_{in}R}{R - i X_C} \nonumber  \\
&= V_{in}\left[ 1 - \frac{V_{in}R}{R - i X_C} \right]
\end{align}
which leads to a ratio of the output voltage to the input voltage of
\begin{align}
\frac{V_{out}}{V_{in}} &= 1 - \frac{R}{R - i X_C} \nonumber \\
&= \frac{-i X_C}{R - i X_C} \nonumber \\
&= \frac{-i X_C(R+iX_C)}{R^2+{X_C}^2} \label{eq:RC_volt_ratio}
\end{align}
Now if we let
\begin{equation}
u = \frac{R}{X_c} = \omega R C \label{eq:RC_u}
\end{equation}
we can see that $R=uX_C$, and putting this in equation \ref{eq:RC_volt_ratio} leads to
\begin{align}
\frac{V_{out}}{V_{in}} &= \frac{-i {X_C}^2(u+i)}{u^2{X_C}^2+{X_C}^2} \\
&= \frac{1-iu}{1+u^2} \label{eq:RC_volt_ratio_final}
\end{align}

We can work out the magnitude and the phase angle of the attenuation through the
filter as follows.
\begin{align}
  \left| \frac{V_{out}}{V_{in}}\right| &= \sqrt{\frac{1-iu}{1+u^2}\frac{1+iu}{1+u^2}}\nonumber \\
 &= \frac{\sqrt{1+u^2}}{1+u^2} \nonumber \\
 &= \frac{1}{\sqrt{1+u^2}} \label{eq:RC_mag}
\end{align}
and for the phase factor
\begin{align}
  \phi &= \arctan\left(\frac{\frac{-u}{1+u^2}}{\frac{1}{1+u^2}}\right) \nonumber \\
  &= -\arctan u \label{eq:RC_phase}
\end{align}

\begin{framed}
\subsection*{Summary}
We looked at the classic example of a low pass RC filter circuit, and discovered
the relationship between the voltage into the filter and the voltage out of the
filter is given by
\begin{equation}
  \frac{V_{out}}{V_{in}} = \frac{1-iu}{1+u^2} \nonumber
\end{equation}
Or in terms of magnitude and phase angle
\begin{align}
  \left|\frac{V_{out}}{V_{in}}\right| &= \frac{1}{\sqrt{1+u^2}} \nonumber \\
  \phi &= - \arctan u \nonumber
\end{align}
where $u=\frac{R}{X_C}=\omega R C$.
\end{framed}

\subsection{Cutoff Frequency}
Letting the attenuation $a=\left| \frac{V_{out}}{V_{in}}\right|$ and rearranging will give us
\begin{align}
  a &= \frac{1}{\sqrt{1+u^2}} \nonumber \\
  1+u^2 &= \frac{1}{a^2} \nonumber \\
  u &= \frac{\sqrt{1-a^2}}{a} \label{eq:RC_1}
\end{align}
Equation \ref{eq:RC_1} and \ref{eq:RC_u} can be used together to calculate component values if a particular attenuation is required at a particular frequency. However an interesting result is the attenuation when $u=1$.
\begin{align}
  u & = 1 \nonumber \\
  a &= \frac{1}{\sqrt{1+u^2}} \nonumber \\
   &= \frac{1}{\sqrt{1+1}} \nonumber \\
  &= \frac{1}{\sqrt{2}} \label{eq:RC_cuttoff_attenuation}
\end{align}
When looking at equation \ref{eq:RC_u} and considering what it means when $u=1$, you will realise this is when the resistance of the capacitor and the
reactance of the capacitor are equal. Also the phase as given by equation \ref{eq:RC_phase} is $\phi=-\arctan{1}=-\frac{\pi}{4}=-\SI{45}{\degree}$

The frequency when $u=1$ is known as the cutoff frequency of the filter, and
is calculated as follows
\begin{align}
  u &= 1 \nonumber \\
  2\pi R C f &= 1 \nonumber \\
  f &= \frac{1}{2\pi R C} \label{eq:RC_cuttof}
\end{align}
