\subsection{ Filter Attenuation }
A simple high pass filter can be formed from a resistor and an inductor in series.

\begin{figure}[h]
  \centering
\begin{circuitikz}
\draw (0,0) to [short,*-,l=$V_{in}$] (1,0)
  to [european resistor, l=$R$] (3,0)
  to [american inductor,l=$L$] (3,-2)
  to (3,-2) node[ground]{};
\draw (3,0) to [short,-*,l=$V_{out}$] (5,0);
\end{circuitikz}
\caption{Circuit diagram for a RL high pass filter.} \label{fig:RL_high_pass_circuit}
\end{figure}
We can see from figure \ref{fig:RL_high_pass_circuit}, that the circuit forms a
potential divider just with a reactive element instead of purely resistive. The
attenuation is then given by the standard potential divider result
\begin{equation}
  \frac{V_{\text{out}}}{V_{\text{in}}} = \frac{i X_{\text{L}}}{R+iX_{\text{L}}} \label{eq:RL_attenuation_1}
\end{equation}
\subsubsection{Cutoff Frequency}
Let's introduce a new variable called $u$, where
\begin{align}
u&=\frac{R}{X_{\text{L}}} \nonumber \\
&= \frac{R}{\omega L} \label{eq:RL_u}
\end{align}
where $\omega = 2 \pi f$. If we look at the frequency when the resulting $u=1$,
which we will label $f_0$ or $\omega_0$
\begin{align}
\frac{R}{\omega_0 L} & = 1 \nonumber \\
\omega_0 &= \frac{R}{L} \label{eq:RL_cutoff_freq}
\end{align}
We call the frequency when $u=1$ the \emph{cutoff frequency}, for reasons that
will be clear later on. This frequency is when the resistance of the resistor is
equal to the reactance of the inductor\footnote{by equal here, we mean the magnitudes are equal. If not the phase shift.}
You can see that we can use the cutoff frequency as a replacement for our $\frac{R}{L}$ value, in equation \ref{eq:RL_u}.
\begin{align}
u&=\frac{R}{\omega L}\nonumber \\
 &= \frac{\omega_0}{\omega} = \frac{f_0}{f} \label{eq:RL_u_f}
\end{align}

\subsubsection{Attenuation revisted}
Now we have some understanding of the variable we introduced $u$, we can substitute it into our equation for the attenuation (equation \ref{eq:RL_attenuation_1}), by noting that from equation \ref{eq:RL_u} $R=u X_{\text{C}}$
\begin{align}
  \frac{V_{\text{out}}}{V_{\text{in}}} & = \frac{i X_{\text{L}}}{R+iX_{\text{L}}}\nonumber \\
   & = \frac{i X_{\text{L}}}{u X_{\text{L}}+iX_{\text{L}}} \nonumber \\
   & = \frac{i }{u +i} \nonumber \\
   & = \frac{1+iu}{u^2 + 1} \label{eq:RL_attenuation_2}
\end{align}

Normally, we don't consider the attenuation as a complex value, instead we care more
about the magnitude and phase shift of an attenuation.
\begin{align}
  \left|\frac{V_{\text{out}}}{V_{\text{in}}} \right| & = \frac{\sqrt{1+u^2}}{1+u^2}\nonumber \\
  & = \frac{1}{\sqrt{1+u^2}} \nonumber \\
  & = \frac{1}{\sqrt{1+\left(\frac{f_0}{f}\right)^2}}\label{eq:RL_attenuation_mag}
\end{align}
Where we have used equation \ref{eq:RL_u_f} in place of $u$. For the phase shift of the filter,
\begin{equation}
  \phi = \arctan{u} = \arctan{\frac{f_0}{f}}
\end{equation}

\begin{framed}
\subsubsection*{Sumary}
In the last section we discovered the cutoff frequency was given by
\begin{equation*}
 f_0 = \frac{1}{2\pi}\frac{R}{L}
\end{equation*}
and that the ratio of resistance to reactance can be given by
\begin{equation*}
  u = \frac{R}{X_{\text{L}}} = \frac{1}{2\pi f} \frac{R}{L} = \frac{f_0}{f}
\end{equation*}
and that the attenuation of the filter is given by
\begin{align*}
  \frac{V_{\text{out}}}{V_{\text{in}}} & = \frac{i X_{\text{L}}}{R+iX_{\text{L}}}\\
   & = \frac{1+i u}{u^2 + 1}
\end{align*}
or in terms of magnitude and phase shift
\begin{align*}
  \left|\frac{V_{\text{out}}}{V_{\text{in}}} \right| & = \frac{1}{\sqrt{1+u^2}} \\
  & = \frac{1}{\sqrt{1+\left(\frac{f_0}{f}\right)^2}}
\end{align*}
\begin{equation*}
  \phi = \arctan{u} =  -\arctan{\frac{f_0}{f}}
\end{equation*}
\end{framed}
