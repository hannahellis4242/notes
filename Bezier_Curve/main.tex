\documentclass[a5paper,12pt]{book}
\usepackage{notes}
\begin{document}
\begin{titlepage}
\begin{center}
\HRule \\[0.5cm]
\textsc{\huge Kinematics}\\[0.5cm]
\HRule \\[0.5cm]
\textsc{\Large Kinematics for particle physics}\\[0.5cm]
\vfill
Hannah Michelle Ellis\\[1.0cm]
\today
\end{center}
\end{titlepage}
\thispagestyle{empty}

\clearpage
\tableofcontents
\listoftables
\listoffigures
\clearpage

\chapter{Bezier Curves}
There are many different kinds of Bezier curve, however they all share some similar properties and can be characterised by their order.
For example a zeroth order Bezier curve is just a point, first order is a line, second order is a quadratic curve and third order is a cubic curve.
Higher order Bezier curves do exist but don't tend to get used much. All Bezier curves have parameters that are given by "control points". The number of control points is one more than the order of the curve.
\section{Examples}
\subsection{Zeroth Order}
\begin{figure}[h]
    \centering
    \definecolor{ududff}{rgb}{0.30196078431372547,0.30196078431372547,1}
\begin{tikzpicture}[line cap=round,line join=round,>=triangle 45,x=1cm,y=1cm]
    \begin{scriptsize}
        \draw [fill=ududff] (2,3) circle (2.5pt);
        \draw[color=ududff] (1.5,3.4) node {$A$};
    \end{scriptsize}
    %\end{axis}
\end{tikzpicture}
    \caption{Example of a zeroth order Bezier curve} \label{fig:example-zeroth}
\end{figure}
The zeroth order Bezier curve is just a point. It has just one control point which is just where the point is positioned. The equation for this curve is very simple, as shown in \ref{eq:zeroth-order}.
\begin{equation} \label{eq:zeroth-order}
    C_{0}(\vec{A})=1\times\vec{A}
\end{equation}
\subsection{First Order}
\begin{figure}[h]
    \centering
    \definecolor{ududff}{rgb}{0.30196078431372547,0.30196078431372547,1}
\begin{tikzpicture}[line cap=round,line join=round,>=triangle 45,x=1cm,y=1cm]
    \draw [line width=2pt] (0,0)-- (3,2);
    \begin{scriptsize}
        \draw [fill=ududff] (0,0) circle (2.5pt);
        \draw[color=ududff] (0,0.5) node {$A$};
        \draw [fill=ududff] (3,2) circle (2.5pt);
        \draw[color=ududff] (3,2.5) node {$B$};
    \end{scriptsize}
\end{tikzpicture}
    \caption{Example of a first order Bezier curve} \label{fig:example-first}
\end{figure}
The first order Bezier curve is a staight line between two control points. You might also see this called a linear interpolation or sometimes lerp. The basic equation for this line is shown in \ref{eq:lerp}.
\begin{equation} \label{eq:lerp}
    C_{1}(\vec{A},\vec{B},t)=(1-t)\vec{A}+t\vec{B}
\end{equation}
\subsection{Second Order}
\begin{figure}[h]
    \centering
    %<<<<<<<WARNING>>>>>>>
% PGF/Tikz doesn't support the following mathematical functions:
% cosh, acosh, sinh, asinh, tanh, atanh,
% x^r with r not integer

% Plotting will be done using GNUPLOT
% GNUPLOT must be installed and you must allow Latex to call external
% programs by adding the following option to your compiler
% shell-escape    OR    enable-write18 
% Example: pdflatex --shell-escape file.tex 

\definecolor{ududff}{rgb}{0.30196078431372547,0.30196078431372547,1}
\begin{tikzpicture}[line cap=round,line join=round,>=triangle 45,x=1cm,y=1cm]
    \clip(-0.74,-0.93) rectangle (4,3);
    \draw[line width=2pt, smooth,samples=100,domain=0:1] plot[parametric] function{(1-t)**(2)*0+2*t*(1-t)*2+t**(2)*3,(1-t)**(2)*0+2*t*(1-t)*2+t**(2)*1};
    \begin{scriptsize}
        \draw [fill=ududff] (0,0) circle (2.5pt);
        \draw[color=ududff] (-0.5,0.48) node {$A$};
        \draw [fill=ududff] (3,1) circle (2.5pt);
        \draw[color=ududff] (3.16,1.42) node {$C$};
        \draw [fill=ududff] (2,2) circle (2.5pt);
        \draw[color=ududff] (2.16,2.42) node {$B$};
    \end{scriptsize}
\end{tikzpicture}
    \caption{Example of a second order Bezier curve} \label{fig:example-second}
\end{figure}
\subsection{Third Order}
\begin{figure}[h]
    \centering
    %<<<<<<<WARNING>>>>>>>
% PGF/Tikz doesn't support the following mathematical functions:
% cosh, acosh, sinh, asinh, tanh, atanh,
% x^r with r not integer

% Plotting will be done using GNUPLOT
% GNUPLOT must be installed and you must allow Latex to call external
% programs by adding the following option to your compiler
% shell-escape    OR    enable-write18 
% Example: pdflatex --shell-escape file.tex 

\definecolor{ududff}{rgb}{0.30196078431372547,0.30196078431372547,1}
\begin{tikzpicture}[line cap=round,line join=round,>=triangle 45,x=1cm,y=1cm]
    \clip(-0.6,-0.1) rectangle (5,5.1);
    \draw[line width=2pt, smooth,samples=100,domain=0:1] plot[parametric] function{3*t*(1-t)**(2)*0.5+3*t**(2)*(1-t)*3+t**(3)*4,3*t*(1-t)**(2)*4.5+3*t**(2)*(1-t)*2+t**(3)*4};
    \begin{scriptsize}
        \draw [fill=ududff] (0,0) circle (2.5pt);
        \draw[color=ududff] (-0.5,0.48) node {$A$};
        \draw [fill=ududff] (3,2) circle (2.5pt);
        \draw[color=ududff] (3.16,2.42) node {$C$};
        \draw [fill=ududff] (0.5,4.5) circle (2.5pt);
        \draw[color=ududff] (0.66,4.92) node {$B$};
        \draw [fill=ududff] (4,4) circle (2.5pt);
        \draw[color=ududff] (4.16,4.42) node {$D$};
    \end{scriptsize}
\end{tikzpicture}
    \caption{Example of a third order Bezier curve} \label{fig:example-third}
\end{figure}
\clearpage
\addcontentsline{toc}{chapter}{Index}
\printindex
\end{document}