\documentclass[a5paper,12pt]{book}
\usepackage{notes}
\begin{document}
\begin{titlepage}
\begin{center}
\HRule \\[0.5cm]
\textsc{\huge Kinematics}\\[0.5cm]
\HRule \\[0.5cm]
\textsc{\Large Kinematics for particle physics}\\[0.5cm]
\vfill
Hannah Michelle Ellis\\[1.0cm]
\today
\end{center}
\end{titlepage}
\thispagestyle{empty}

\clearpage
\tableofcontents
\listoftables
\listoffigures
\clearpage

\chapter{Bezier Curves}
There are many different kinds of Bezier curve, however they all share some similar properties and can be characterised by their order.
For example a zeroth order Bezier curve is just a point, first order is a line, second order is a quadratic curve and third order is a cubic curve.
Higher order Bezier curves do exist but don't tend to get used much. All Bezier curves have parameters that are given by "control points". The number of control points is one more than the order of the curve.
\section{Examples}

\clearpage
\addcontentsline{toc}{chapter}{Index}
\printindex
\end{document}