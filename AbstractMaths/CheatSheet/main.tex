\documentclass[12pt,a4paper,twocolumn]{article}
\usepackage[utf8]{inputenc}
\usepackage[top=1.5cm, bottom=1.5cm, left=1.5cm, right=1.5cm]{geometry}
\usepackage{amsmath}
\usepackage{graphicx}
\usepackage{color}
\usepackage{footnote}
\numberwithin{equation}{section}
\newcommand{\HRule}{\rule{\linewidth}{0.5mm}}
\usepackage{makeidx}
\makeindex
\usepackage{amssymb}
\usepackage{bm}
\newcommand{\uveci}{{\bm{\hat{\textnormal{\bfseries\i}}}}}
\newcommand{\uvecj}{{\bm{\hat{\textnormal{\bfseries\j}}}}}
\newcommand{\uveck}{{\bm{\hat{\bm{k}}}}}
\usepackage{framed}
\usepackage[utf8]{inputenc}
\usepackage{pgf,tikz}
\usetikzlibrary{arrows}
\usepackage{cancel}
\usepackage{xcolor}
\newcommand\Ccancel[2][black]{\renewcommand\CancelColor{\color{#1}}\cancel{#2}}
\usepackage{physics}

\title{Abstract Maths Cheat Sheet}
\date{}
\author{}
\begin{document}
\maketitle

\section{Ket-vector Axioms}
\begin{enumerate}
    \item The sum of any two ket-vectors is also a ket-vector:
    \begin{equation}
        \ket{A}  + \ket{B} = \ket{C}
    \end{equation}
    \item Vector addition is commutative:
    \begin{equation}
        \ket{ A } +\ket{ B } =\ket{B} +\ket{A}
    \end{equation}
    \item Vector addition is associative:
    \begin{equation}
        \left(\ket{ A } +\ket{ B }\right)+\ket{ C } =\ket{ A } +  \left(\ket{B }+\ket{ C }\right)
    \end{equation}
    \item There is a unique vector $0$ such that when you add it to any ket, it gives the same ket back:
    \begin{equation}
       \ket{ A } + 0 =\ket{ A }
    \end{equation}
    \item Given any ket $\ket{A}$, there is a unique ket $-\ket{A}$ such that
    \begin{equation}
       \ket{A} + \left( -\ket{A} \right)
    \end{equation}
    \item Given any $\ket{A}$ and any complex number $z$, you can multiply them to get a new ket. Also, multiplication by a scalar is linear:
    \begin{equation}
       \ket{zA} = z\ket{ A } =\ket{B}
    \end{equation}
    \item The distributive property holds:
    \begin{align}
        z\left(\ket{A} +\ket{B}\right) &=  z\ket{A} + z\ket{B}\\
        \left( z+w\right)\ket{A} &= z\ket{A} + w\ket{A}
    \end{align}
\end{enumerate}

\section{Bra-vector Axioms}
\begin{enumerate}
    \item The sum of any two bra-vectors is also a bra-vector:
    \begin{equation}
        \bra{A}  + \bra{B} = \bra{C}
    \end{equation}
    \item Vector addition is commutative:
    \begin{equation}
        \bra{ A } +\bra{ B } =\bra{B} +\bra{A}
    \end{equation}
    \item Vector addition is associative:
    \begin{equation}
        \left(\bra{ A } +\bra{ B }\right)+\bra{ C } =\bra{ A } +  \left(\bra{B }+\bra{ C }\right)
    \end{equation}
    \item There is a unique vector $0$ such that when you add it to any bra, it gives the same bra back:
    \begin{equation}
       \bra{ A } + 0 =\bra{ A }
    \end{equation}
    \item Given any bra $\bra{A}$, there is a unique bra $-\bra{A}$ such that
    \begin{equation}
       \bra{A} + \left( -\bra{A} \right)
    \end{equation}
    \item Given any $\bra{A}$ and any complex number $z$, you can multiply them to get a new bra. Also, multiplication by a scalar is linear:
    \begin{equation}
       \bra{zA} = z\bra{ A } =\bra{B}
    \end{equation}
    \item The distributive property holds:
    \begin{align}
        z\left(\bra{A} +\bra{B}\right) &=  z\bra{A} + z\bra{B}\\
        \left( z+w\right)\bra{A} &= z\bra{A} + w\bra{A}
    \end{align}
\end{enumerate}

\end{document}
