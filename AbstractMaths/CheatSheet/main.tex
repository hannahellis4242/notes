\documentclass[12pt,a4paper,twocolumn]{article}
\usepackage{notes}

\title{Abstract Maths Cheat Sheet}
\date{}
\author{}
\begin{document}
\maketitle

\section{Ket-vector Axioms}
\begin{enumerate}
    \item The sum of any two ket-vectors is also a ket-vector:
    \begin{equation}
        \ket{A}  + \ket{B} = \ket{C}
    \end{equation}
    \item Vector addition is commutative:
    \begin{equation}
        \ket{ A } +\ket{ B } =\ket{B} +\ket{A}
    \end{equation}
    \item Vector addition is associative:
    \begin{equation}
        \left(\ket{ A } +\ket{ B }\right)+\ket{ C } =\ket{ A } +  \left(\ket{B }+\ket{ C }\right)
    \end{equation}
    \item There is a unique vector $0$ such that when you add it to any ket, it gives the same ket back:
    \begin{equation}
       \ket{ A } + 0 =\ket{ A }
    \end{equation}
    \item Given any ket $\ket{A}$, there is a unique ket $-\ket{A}$ such that
    \begin{equation}
       \ket{A} + \left( -\ket{A} \right)
    \end{equation}
    \item Given any $\ket{A}$ and any complex number $z$, you can multiply them to get a new ket. Also, multiplication by a scalar is linear:
    \begin{equation}
       \ket{zA} = z\ket{ A } =\ket{B}
    \end{equation}
    \item The distributive property holds:
    \begin{align}
        z\left(\ket{A} +\ket{B}\right) &=  z\ket{A} + z\ket{B}\\
        \left( z+w\right)\ket{A} &= z\ket{A} + w\ket{A}
    \end{align}
\end{enumerate}
Taken together 6 and 7 are often call linearity

\section{Bra-vector Axioms}
\begin{enumerate}
    \item The sum of any two bra-vectors is also a bra-vector:
    \begin{equation}
        \bra{A}  + \bra{B} = \bra{C}
    \end{equation}
    \item Vector addition is commutative:
    \begin{equation}
        \bra{ A } +\bra{ B } =\bra{B} +\bra{A}
    \end{equation}
    \item Vector addition is associative:
    \begin{equation}
        \left(\bra{ A } +\bra{ B }\right)+\bra{ C } =\bra{ A } +  \left(\bra{B }+\bra{ C }\right)
    \end{equation}
    \item There is a unique vector $0$ such that when you add it to any bra, it gives the same bra back:
    \begin{equation}
       \bra{ A } + 0 =\bra{ A }
    \end{equation}
    \item Given any bra $\bra{A}$, there is a unique bra $-\bra{A}$ such that
    \begin{equation}
       \bra{A} + \left( -\bra{A} \right)
    \end{equation}
    \item Given any $\bra{A}$ and any complex number $z$, you can multiply them to get a new bra. Also, multiplication by a scalar is linear:
    \begin{equation}
       \bra{zA} = z\bra{ A } =\bra{B}
    \end{equation}
    \item The distributive property holds:
    \begin{align}
        z\left(\bra{A} +\bra{B}\right) &=  z\bra{A} + z\bra{B}\\
        \left( z+w\right)\bra{A} &= z\bra{A} + w\bra{A}
    \end{align}
\end{enumerate}
Taken together 6 and 7 are often call linearity
\section{Bras and Kets}
As we have seen, the complex numbers have a dual version: in the form of complex conjugate numbers. In the same way, a complex vector space has a dual version that is essentially the complex conjugate vector space.

For every ket-vector $\ket{A}$, there is a bra-vector in the dual space, denoted by $\bra{A}$. Bra and Ket vectors together form inner products of bras and kets, using expressions like <B|A> to form bra-kets or brackets.

Inner products are extremely important in the mathematical machinery of quantum mechanics,and for characterizing vector spaces in general.

Bra-vectors satisfy the same axioms as the ket-vectors, but there are two things to keep in mind about the correspondence between kets and bras:

    Suppose <A| is the bra corresponding to the ket |A>, and <B| is the bra corresponding to the ket |B>. Then the bra corresponding to |A> + |B> is <A| + <B|

    If z is a complex number, then it is not true that the bra corresponding to z|A> is <A|z. You have to remember to complex-conjugate. Thus, the bra corresponding to z|A> is <A|z*

\section{Inner Products}
\begin{enumerate}
\item placeholder
\end{enumerate}
\end{document}
